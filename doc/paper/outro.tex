\section{Conclusions}

The contributions of this work are:
\begin{itemize}

\item an implementation of the ideas in $L^3$~\cite{l3} to mixed
  linear/non-linear programming~\cite{mixed}, where the
  languages being mixed are Rust and JavaScript,

\item a treatment of garbage collection (rather than region-based
  memory management) for such a system, and

\item a treatment of how operations such as accessing, mutating, and
  rooting can change the lifetimes of objects.
  
\end{itemize}
The main item left for future work is formalizing the approach
described here: memory safety is conjectured, but not proved
formally.

There are some aspects of the API which need more investigation:
\begin{itemize}

\item Other JavaScript engines take a different approach to
  rooting, notably V8 \emph{handle scopes}~\cite{v8-embedding},
  which have different trade-offs. In terms of this paper, the
  roots are attached to the JS context, rather than stored
  on the stack. It would be interesting to compare these approaches.

\item Josephine uses phantom types to track which compartment
  memory is allocated in, but does not support features such
  as \emph{cross-compartment wrappers}~\cite{compartments},
  which allow references between compartments.

\item In this paper, we have just used the SpiderMonkey runtime
  engine for its garbage collector, but it is a full-featured
  JavaScript engine, and it would be good to provide safe
  access to executing JS code. This would be simpler to achieve
  if there were a JS type system to generate bindings from,
  such as TypeScript~\cite{typescript}.
  
\end{itemize}
The distribution includes some simple examples such as doubly-linked lists
and a stripped-down DOM, but more examples are needed to see if the API
is usable for practical code.
