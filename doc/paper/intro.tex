\section{Introduction}

This paper discusses the Josephine API~\cite{josephine} for using
the Spidermonkey~\cite{spidermonkey} JavaScript runtime to safely
manage the lifetime of Rust~\cite{rust} data. That sentence needs
some unpacking.

\subsection{Rust}

Rust is a systems programming language which uses fancy types to
ensure memory safety even in the presence of mutable update, and the
absence of a garbage collector. Rust has an affine type system, which
allows data to be discarded but does not allow data to be arbitrarily
copied. For example, the Rust program:
\begin{verbatim}
  let hello = String::from("hello");
  let moved = hello;
  println!("Oh look {} is hello", moved);
\end{verbatim}
is fine, but the program:
\begin{verbatim}
  let hello = String::from("hello");
  let copied = hello;
  println!("Oh look {} is {}", hello, copied);
\end{verbatim}
is not, since \verb|hello| and \verb|copied| are simultaneously live. Trying to compile
this program produces an error:
\begin{verbatim}
use of moved value: `hello`
 --> src/main.rs:4:32
  |
3 |   let copied = hello;
  |       ------ value moved here
4 |   println!("Oh look {} is {}", hello, copied);
  |                                ^^^^^ value used here after move
\end{verbatim}
The use of affine types allows aliasing to be tracked. For example, a
classic problem with aliasing is appending a string to itself. In
Rust, an example of appending a string is:
\begin{verbatim}
  let mut hello = String::from("hello");
  let ref world = String::from("world");
  hello.push_str(world);
  println!("Oh look hello is {}", hello);
\end{verbatim}
The important operation is \verb|hello.push_str(world)|, which mutates the
string \verb|hello| (hence the \verb|mut| annotation on the declaration of \verb|hello|).
The appended string \verb|world| is passed by reference,
(hence the \verb|ref| annotation on the declaration of \verb|world|).

A problem with mutably appending strings is ensuring that the string
is not appended to itself, for example the documentation for
C \verb|strcat|~\cite{strcat} states ``Source and destination may not
overlap'', but C does not check aliasing and relies on the programmer
to ensure correctness. In contrast, attempting to append a string
to itself in Rust:
\begin{verbatim}
  let ref mut hello = String::from("hello");
  hello.push_str(hello);
\end{verbatim}
produces an error:
\begin{verbatim}
cannot borrow `*hello` as immutable because it is also borrowed as mutable
 --> src/main.rs:3:18
  |
3 |   hello.push_str(hello);
  |   -----          ^^^^^- mutable borrow ends here
  |   |              |
  |   |              immutable borrow occurs here
  |   mutable borrow occurs here
\end{verbatim}
In Rust, the crucial invariant maintained by affine types is:
\begin{quote}\em
  Any memory that can be reached simultaneously by two different paths
  is immutable.
\end{quote}
For example in \verb|hello.push(hello)| there are two occurrences of \verb|hello| that
are live simultaneously, the first of which is mutating the string, so this is outlawed.

In order to track statically which variables are live simultaneously, Rust uses a lifetime
system similar to that used by region-based memory~\cite{regions}. Each allocation of
memory has a lifetime $\alpha$, and lifetimes are ordered $\alpha\subseteq\beta$.
Each code block introduces a lifetime, and for data which does not escape from its scope,
the nesting of blocks determines the ordering of lifetimes.

For example in the program:
\begin{verbatim}
  let ref x = String::from("hi");
  {
    let ref y = x;
    println!("y is {}", y);
  }
  println!("x is {}", x);
\end{verbatim}
the variable \verb|x| has a lifetime $\alpha$ given by the outer block,
and the variable \verb|y| has a lifetime $\beta\subseteq\alpha$ given by the inner block.

These lifetimes are mentioned in the types of references: the type $\REF\alpha T$
is a reference giving immutable access to data of type $T$, which will live at least as long as
$\alpha$. Similarly, the type $\REFMUT\alpha T$ gives mutable access to the data: the crucial
difference is that $\REF\alpha T$ is a copyable type, but $\REFMUT\alpha T$ is not.
For example
the type of \verb|x| is $\REF\alpha\STRING$ and the type of \verb|y| is
$\REF\beta(\REF\alpha\STRING)$, which is well-formed because $\beta\subseteq\alpha$.

Lifetimes are used to prevent another class of memory safety issues: use-after-free.
For example, consider the program:
\begin{verbatim}
  let hi = String::from("hi");
  let ref mut handle = &hi;
  {
    let lo = String::from("lo");
    *handle = &lo;
  } // lo is deallocated here
  println!("handle is {}", **handle);
\end{verbatim}
If this program were to execute, its behaviour would be undefined,
since \verb|**handle| is used after \verb|lo|
(which \verb|handle| points to) is deallocated. Fortunately, this program
does not pass Rust's borrow-checker:
\begin{verbatim}
`lo` does not live long enough
 --> src/main.rs:6:11
  |
6 |     *handle = &lo;
  |                ^^ borrowed value does not live long enough
7 |   } // lo is deallocated here
  |   - `lo` dropped here while still borrowed
8 |   println!("handle is {}", **handle);
9 | }
  | - borrowed value needs to live until here
\end{verbatim}
This use-after-free error can be detected because (naming the outer lifetime to be
$\alpha$ and the inner lifetime to be $\beta\subseteq\alpha$) \verb|handle| has type
$\REFMUT\alpha\REF\alpha\STRING$, but \verb|&lo| only has type $\REF\beta\STRING$, no
$\REF\alpha\STRING$ as required by the assignment.

Lifetimes avoid use-after-free by maintaining two invariants:
\begin{quote}\em
  Any dereference happens during the lifetime of the reference, \\
  and deallocation happens after the lifetime of all references.
\end{quote}
There is more to the Rust type system than described here
(higher-order functions, traits, variance, \dots) but the important features
are \emph{affine types} and \emph{lifetimes} for ensuring memory safety,
even in the presence of manual memory management.
