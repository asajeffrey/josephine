\section{Introduction}

This paper discusses the Josephine API~\cite{josephine} for using
the Spidermonkey~\cite{spidermonkey} JavaScript runtime to safely
manage the lifetime of Rust~\cite{rust} data. That sentence needs
some unpacking.

\subsection{Rust}

Rust is a systems programming language which uses fancy types to
ensure memory safety even in the presence of mutable update, and the
absence of a garbage collector. Rust has an affine type system, which
allows data to be discarded but does not allow data to be arbitrarily
copied. For example, the Rust program:
\begin{verbatim}
  let hello = String::from("hello");
  let moved = hello;
  println!("Oh look {} is hello", moved);
\end{verbatim}
is fine, but the program:
\begin{verbatim}
  let hello = String::from("hello");
  let copied = hello;
  println!("Oh look {} is {}", hello, copied);
\end{verbatim}
is not, since \verb|hello| and \verb|copied| are simultaneously live. Trying to compile
this program produces an error:
\begin{verbatim}
use of moved value: `hello`
 --> src/main.rs:4:32
  |
3 |   let copied = hello;
  |       ------ value moved here
4 |   println!("Oh look {} is {}", hello, copied);
  |                                ^^^^^ value used here after move
\end{verbatim}
The use of affine types allows aliasing to be tracked. For example, a
classic problem with aliasing is appending a string to itself. In
Rust, an example of appending a string is:
\begin{verbatim}
  let mut hello = String::from("hello");
  let ref world = String::from("world");
  hello.push_str(world);
  println!("Oh look hello is {}", hello);
\end{verbatim}
The important operation is \verb|hello.push_str(world)|, which mutates the
string \verb|hello| (hence the \verb|mut| annotation on the declaration of \verb|hello|).
The appended string \verb|world| is passed by reference,
(hence the \verb|ref| annotation on the declaration of \verb|world|).

A problem with mutably appending strings is ensuring that the string
is not appended to itself, for example the documentation for
C \verb|strcat|~\cite{strcat} states ``Source and destination may not
overlap'', but C does not check aliasing and relies on the programmer
to ensure correctness. In contrast, attempting to append a string
to itself in Rust:
\begin{verbatim}
  let ref mut hello = String::from("hello");
  hello.push_str(hello);
\end{verbatim}
produces an error:
\begin{verbatim}
cannot borrow `*hello` as immutable because it is also borrowed as mutable
 --> src/main.rs:3:18
  |
3 |   hello.push_str(hello);
  |   -----          ^^^^^- mutable borrow ends here
  |   |              |
  |   |              immutable borrow occurs here
  |   mutable borrow occurs here
\end{verbatim}
