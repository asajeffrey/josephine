\section{Introduction}

This paper is about the interface between languages
which use a garbage collector and those which use fancy
types for safe manual memory management.

Garbage collection is the most common memory management technique for
functional programming languages, dating back to LISP~\cite{LISP}.
Having a garbage collector guarantees memory safety, but at the
cost of having a required runtime system.

Imperative languages often require the programmer to perform
manual memory management, such as the \verb|malloc| and \verb|free|
functions provided by C~\cite{K+R}. The safety of a program
(in particular the absence of \emph{use-after-free} errors)
is considered the programmer's problem.
More recently, laguages such as Cyclone~\cite{cyclone}
and Rust~\cite{rust} have used fancy type systems
such as substructural types~\cite{girard,Go4,walker}
and region analysis~\cite{regions} to guarantee memory
safety without garbage collection.

This paper discusses the Josephine API~\cite{josephine} for using the
garbage collector provided by the Spidermonkey~\cite{spidermonkey}
JavaScript runtime to safely manage the lifetime of Rust~\cite{rust}
data. It uses techniques from $L^3$~\cite{l3} and its application
to regions~\cite{l3-with-regions}, but the application to languages
with a tracing garbage collector is new.

\subsection{Rust}

Rust is a systems programming language which uses fancy types to
ensure memory safety even in the presence of mutable update, and
manual memory management. Rust has an affine type system, which
allows data to be discarded but does not allow data to be arbitrarily
copied. For example, the Rust program:
\begin{verbatim}
  let hello = String::from("hello");
  let moved = hello;
  println!("Oh look {} is hello", moved);
\end{verbatim}
is fine, but the program:
\begin{verbatim}
  let hello = String::from("hello");
  let copied = hello;
  println!("Oh look {} is {}", hello, copied);
\end{verbatim}
is not, since \verb|hello| and \verb|copied| are simultaneously live. Trying to compile
this program produces:
\begin{verbatim}
  use of moved value: `hello`
   --> src/main.rs:4:32
    |
  3 |   let copied = hello;
    |       ------ value moved here
  4 |   println!("Oh look {} is {}", hello, copied);
    |                                ^^^^^ value used here after move
\end{verbatim}
The use of affine types allows aliasing to be tracked. For example, a
classic problem with aliasing is appending a string to itself. In
Rust, an example of appending a string is:
\begin{verbatim}
  let mut hello = String::from("hello");
  let ref world = String::from("world");
  hello.push_str(world);
  println!("Oh look hello is {}", hello);
\end{verbatim}
The important operation is \verb|hello.push_str(world)|, which mutates the
string \verb|hello| (hence the \verb|mut| annotation on the declaration of \verb|hello|).
The appended string \verb|world| is passed by reference,
(hence the \verb|ref| annotation on the declaration of \verb|world|).

A problem with mutably appending strings is ensuring that the string
is not appended to itself, for example the documentation for
C \verb|strcat| states ``Source and destination may not
overlap'', but C does not check aliasing and relies on the programmer
to ensure correctness. In contrast, attempting to append a string
to itself in Rust:
\begin{verbatim}
  let ref mut hello = String::from("hello");
  hello.push_str(hello);
\end{verbatim}
produces an error:
\begin{verbatim}
  cannot borrow `*hello` as immutable because it is also borrowed as mutable
   --> src/main.rs:3:18
    |
  3 |   hello.push_str(hello);
    |   -----          ^^^^^- mutable borrow ends here
    |   |              |
    |   |              immutable borrow occurs here
    |   mutable borrow occurs here
\end{verbatim}
In Rust, the crucial invariant maintained by affine types is:
\begin{quote}\em
  Any memory that can be reached simultaneously by two different paths
  is immutable.
\end{quote}
For example in \verb|hello.push(hello)| there are two occurrences of \verb|hello| that
are live simultaneously, the first of which is mutating the string, so this is outlawed.

In order to track statically which variables are live simultaneously, Rust uses a lifetime
system similar to that used by region-based memory~\cite{regions}. Each allocation of
memory has a lifetime $\alpha$, and lifetimes are ordered $\alpha\subseteq\beta$.
Each code block introduces a lifetime, and for data which does not escape from its scope,
the nesting of blocks determines the ordering of lifetimes.

For example in the program:
\begin{verbatim}
  let ref x = String::from("hi");
  {
    let ref y = x;
    println!("y is {}", y);
  }
  println!("x is {}", x);
\end{verbatim}
the variable \verb|x| has a lifetime $\alpha$ given by the outer block,
and the variable \verb|y| has a lifetime $\beta\subseteq\alpha$ given by the inner block.

These lifetimes are mentioned in the types of references: the type $\REF\alpha T$
is a reference giving immutable access to data of type $T$, which will live at least as long as
$\alpha$. Similarly, the type $\REFMUT\alpha T$ gives mutable access to the data: the crucial
difference is that $\REF\alpha T$ is a copyable type, but $\REFMUT\alpha T$ is not.
For example
the type of \verb|x| is $\REF\alpha\STRING$ and the type of \verb|y| is
$\REF\beta(\REF\alpha\STRING)$, which is well-formed because $\beta\subseteq\alpha$.

Lifetimes are used to prevent another class of memory safety issues: use-after-free.
For example, consider the program:
\begin{verbatim}
  let hi = String::from("hi");
  let ref mut handle = &hi;
  {
    let lo = String::from("lo");
    *handle = &lo;
  } // lo is deallocated here
  println!("handle is {}", **handle);
\end{verbatim}
If this program were to execute, its behaviour would be undefined,
since \verb|**handle| is used after \verb|lo|
(which \verb|handle| points to) is deallocated. Fortunately, this program
does not pass Rust's borrow-checker:
\begin{verbatim}
  `lo` does not live long enough
   --> src/main.rs:6:11
    |
  6 |     *handle = &lo;
    |                ^^ borrowed value does not live long enough
  7 |   } // lo is deallocated here
    |   - `lo` dropped here while still borrowed
  8 |   println!("handle is {}", **handle);
  9 | }
    | - borrowed value needs to live until here
\end{verbatim}
This use-after-free error can be detected because (naming the outer lifetime to be
$\alpha$ and the inner lifetime to be $\beta\subseteq\alpha$) \verb|handle| has type
$\REFMUT\alpha\REF\alpha\STRING$, but \verb|&lo| only has type $\REF\beta\STRING$, no
$\REF\alpha\STRING$ as required by the assignment.

Lifetimes avoid use-after-free by maintaining two invariants:
\begin{quote}\em
  Any dereference happens during the lifetime of the reference, \\
  and deallocation happens after the lifetime of all references.
\end{quote}
There is more to the Rust type system than described here
(higher-order functions, traits, variance, concurrency, \dots) but the important features
are \emph{affine types} and \emph{lifetimes} for ensuring memory safety,
even in the presence of manual memory management.

\subsection{Spidermonkey}

Spidermonkey is Mozilla's JavaScript runtime, used in the Firefox browser,
and the Servo~\cite{servo} next-generation web engine. This is a full-featured
JS implementation, but the focus of this paper is its automatic memory management.

Inside a web engine, there are often native implementations of HTML features,
which are exposed to JavaScript using DOM interfaces. For example, an HTML image
is exposed to JavaScript as a DOM object representing an \verb|<img>| element,
but behind the scenes there is native code responsible for loading and rendering
images.

When a JavaScript object is garbage collected, a destructor is called to
deallocate any attached native memory. In the case that the native code
is implemented in Rust, this leads to a situation where Rust relies on affine
types and lifetimes for memory safety, but JavaScript respects neither of these.
As a result, the raw Spidermonkey interface to Rust is very unsafe,
for example there are nearly 400 instances of unsafe code in the Servo
DOM implementation:
\begin{verbatim}
  $ grep "unsafe_code" components/script/dom/*.rs | wc
      393     734   25514
\end{verbatim}
Since JavaScript does not respect Rust's affine types,
Servo's DOM implementation makes use of Rust~\cite[\S3.11]{rust}
\emph{interior mutability} which replaces the compile-time type
checks with run-time dynamic checks. This carries run-time overhead,
and the possibility of checks failing, and Servo panicking.

Moreover, Spidermonkey has its own invariants, and if an embedding
application does not respect these invariants, then runtime errors can
occur. One of these invariants is the division of JavaScript memory
into \emph{compartments}, which can be garbage collected
separately. The runtime has a notion of ``current compartment'',
and the embedding application is asked to maintain two invariants:
\begin{itemize}
  \item whenever an object is used, the object is in the current compartment, and
  \item there are no references between objects which cross compartments.
\end{itemize}
In order for native code to interact well with the Spidermonkey garbage collector,
it has to provide two functions:
\begin{itemize}
\item a \emph{trace} function, that given an object, iterates over all of the
  JavaScript objects which are reachable from it, and
\item a \emph{roots} function, which iterates over all of the JavaScript
  objects that are live on the call stack.
\end{itemize}
From these two functions, the garbage collector can find all of the reachable
JavaScript objects, including those reachable from JavaScript directly, and
those reached via native code.

Automatically generating the trace function is reasonably straightforward
metaprogramming, but rooting safely turns out to be quite tricky.
Servo provides an approximate analysis for safe rooting using an ad-hoc
static analysis (the \emph{rooting lint}), but this is problematic because
a) the lint is syntax-driven, so does not understand about Rust features
such as generics, and b) even if it could be made sound it is disabled
more than 200 times:
\begin{verbatim}
  $ grep "unrooted_must_root" components/script/dom/*.rs | wc
      213     456   15961
\end{verbatim}

\subsection{Josephine}

Josephine~\cite{josephine} is intended to act as a safe bridge between
Spidermonkey and Rust. Its goals are:
\begin{itemize}

\item to use JavaScript to manage the lifetime of Rust data,
  and to allow JavaScript to garbage collect unreachable data,

\item to allow references to JavaScript-managed data to be freely copied and discarded,
  relying on Spidermonkey's garbage collector for safety,

\item to maintain Rust memory safety via affine types and lifetimes,
  without requiring additional static analysis such as the rooting lint,

\item to allow mutable and immutable access to Rust data via JavaScript-managed references,
  so avoiding interior mutability, and

\item to provide a rooting API to ensure that JavaScript-managed data is not garbage collected
  while it is being used.

\end{itemize}
Josephine is intended to be safe, in that any programs built using Josephine's API
do not introduce undefined behaviour or runtime errors.
Josephine achieves this by providing controlled access to
Spidermonkey's \emph{JavaScript context}, and maintaining invariants about it:
\begin{itemize}

\item immutable access to JavaScript-managed data requires immutable access
  to the JavaScript context,

\item mutable access to JavaScript-managed data requires mutable access
  to the JavaScript context, and

\item any action that can trigger garbage collection (for example allocating
  new objects) requires mutable access to the JavaScript context.

\end{itemize}
As a result, since access to the JavaScript context does respect
Rust's affine types, mutation or garbage collection cannot occur
simultaneously with accessing JavaScript-managed data.

In other words, Josephine treats the JavaScript context as an affine
access token, or capability, which controls access to the JavaScript-managed
data. The data accesses respect affine types, even though the JavaScript objects
themselves do not.

This use of an access token to safely access data in a substructural
type system is \emph{not} new, it is the heart of Ahmed, Fluet and
Morrisett's $L^3$ Linear Language with Locations~\cite{l3} and its
application to linear regions~\cite{regions}.

Moreover, type systems for mixed linear/non-linear programming have
been known for a long time, for example Benton's 1994~\cite{mixed}.
The aspects that are novel are:
\begin{itemize}

\item the linear language is Rust, and the non-linear language is
  JavaScript, which are both industrial-strength languages,

\item the treatment of garbage collection requires a
  different treatment than regions in $L^3$, which have a stack
  discipline, and

\item the token contains more state in its type, carrying more than
  just read/write access, but the current compartment, and the capability
  to trigger garbage collection.

\end{itemize}

\subsection*{Acknowledgments}

This work benefited greatly from conversations with
Amal Ahmed,
Nick Benton,
Josh Bowman-Matthews,
Manish Goregaokar,
Bobby Holly, and
Anthony Ramine.
