\section{Compartments}

Spidermonkey uses \emph{compartments} to organize memory,
so that garbage collection does not have to sweep the
entire memory, just one compartment\footnote{%
  For purposes of this paper, we are ignoring the distinction between zones and
  compartments.
}
To achieve this, SpiderMonkey maintains the invariant:
\begin{quote}\em
  There are no direct references between compartments.
\end{quote}
This invariant is expected to be maintained by
any native data: tracing a JS-managed object
should never result in tracing an object from a different
compartment.

In Josephine, the compartment that
native data has been placed in is part of its
type. Data of type $\JSManaged{C,T}$ is attached
to a JS object in compartment $C$.

\subsection{Maintaining the invariant}

It would be possible for user-defined types to break
the compartment invariant, for example:
\begin{verbatim}
  type BadCell<'a, C, D> = JSManaged<'a, C, NativeBadCell<'a, C, D>>;
\end{verbatim}
where:
\begin{verbatim}
  struct NativeBadCell<'a, C, D> {
    data: String,
    prev: Option<BadCell<'a, C, C>>,
    next: Option<BadCell<'a, D, D>>,
  }
\end{verbatim}
This type vioated the compartment invariant, because
a cell of type \verb|BadCell<'a, C, D>| is in compartment
\verb|C| but its next pointer is in compartment \verb|D|.

The maintain the compartment invariant, we introduce
a trait similar to \verb|JSLifetime|, but for compartments:
\begin{verbatim}
  pub unsafe trait JSCompartmental<C, D> {
    type ChangeCompartment;
    fn is_in_compartment<S>(&self, _cx: &JSContext<S>) -> bool where
        S: InCompartment<D>;
  }
\end{verbatim}
In the same way that $\JSLifetime{\alpha}$ is used to implement
lifetime substitution $T[\alpha/\beta]$, the trait $\JSCompartmental{C,D}$
is used to implement compartment substitution $T[D/C]$. A type $T$ implementing
$\JSCompartmental{C,D}$ is asked to ensure that:
\begin{itemize}
\item $T$ is in compartment $C$,
\item $T$ only contains references to other types implementing $\JSCompartmental{C,D}$, and
\item $T\cc\ChangeCompartment$ is $T[D/C]$.
\end{itemize}


\subsection{Creating compartments}

\subsection{Entering a compartment}

\subsection{Wilcard compartments}

