\section{Compartments}

SpiderMonkey uses \emph{compartments} to organize memory,
so that garbage collection does not have to sweep the
entire memory, just one compartment\footnote{%
  For purposes of this paper, we are ignoring the distinction between zones and
  compartments.
}
To achieve this, SpiderMonkey maintains the invariant:
\begin{quote}\em
  There are no direct references between compartments.
\end{quote}
This invariant is expected to be maintained by
any native data: tracing a JS-managed object
should never result in tracing an object from a different
compartment.

In Josephine, the compartment that
native data has been placed in is part of its
type. Data of type $\JSManaged{C,T}$ is attached
to a JS object in compartment $C$.

\subsection{Maintaining the invariant}

It would be possible for user-defined types to break
the compartment invariant, for example:
\begin{verbatim}
  type BadCell<'a, C, D> = JSManaged<'a, C, NativeBadCell<'a, C, D>>;
\end{verbatim}
where:
\begin{verbatim}
  struct NativeBadCell<'a, C, D> {
    data: String,
    prev: Option<BadCell<'a, C, C>>,
    next: Option<BadCell<'a, D, D>>,
  }
\end{verbatim}
This type violates the compartment invariant, because
a cell of type \verb|BadCell<'a, C, D>| is in compartment
\verb|C| but its next pointer is in compartment \verb|D|.

To maintain the compartment invariant, we introduce
a trait similar to \verb|JSLifetime|, but for compartments:
\begin{verbatim}
  pub unsafe trait JSCompartmental<C, D> {
    type ChangeCompartment;
  }
\end{verbatim}
In the same way that $\JSLifetime{\alpha}$ is used to implement
lifetime substitution $T[\alpha/\beta]$, the trait $\JSCompartmental{C,D}$
is used to implement compartment substitution $T[D/C]$. A type $T$ implementing
$\JSCompartmental{C,D}$ is asked to ensure that:
\begin{itemize}

\item $T$ is in compartment $C$,
\item $T$ only contains references to other types implementing $\JSCompartmental{C,D}$, and
\item $T\cc\ChangeCompartment$ is $T[D/C]$.

\end{itemize}
If the implementation of this type is incorrect, there may be safety issues,
which is why the trait is marked as \verb|unsafe|. Fortunately, deriving an
implementation of this trait is straightforward meta-programming.
Josephine provides a \verb|#[derive(JSCompartmental)|
which is guaranteed to maintain
the compartment invariant.

\subsection{Creating compartments}

In SpiderMonkey, a new compartment is created each time a
\emph{global object}~\cite[\S18]{ecmascript} is created.

In Josephine, there are two functions: one to create a new compartment,
and another to attach native data to the global. The global object can
be accessed with \verb|cx.global()|. For example:
\begin{verbatim}
  let cx = cx.create_compartment();
  let name = String::from("Alice");
  let cx = cx.global_manage(name);
\end{verbatim}
In some cases, the global (in freshly created compartment $C$)
contains some JS-managed data (also in compartment $C$), which is why
the initialization is split into two steps. First, create compartment
compartment $C$, then initialize the native data, which may make
use of $C$. For example:
\begin{verbatim}
   let cx = cx.create_compartment();
   let ref mut root = cx.new_root();
   let name = cx.manage(String::from("Alice")).in_root(root);
   let cx = cx.global_manage(NativeMyGlobal { name: name });
\end{verbatim}
where:
\begin{verbatim}
  struct NativeMyGlobal<'a, C> { name: JSManaged<'a, C, String> }
  type MyGlobal<'a, C> = JSManaged<'a, C, NativeMyGlobal<'a, C>>;
\end{verbatim}
The type rule for creating a compartment is:
\begin{quote}
  if $cx:\REFMUT{\alpha}{\JSContext{S}}$
  and $S: \CanAlloc + \CanAccess$ \\
  then $cx.\createCompartment() : \JSContext{S'}$ \\\mbox{}\quad
  where $S': \alpha + \CanAlloc + \InCompartment{C} + \IsInitializing{\alpha,C,T}$ \\\mbox{}\qquad
  for fresh $C: \Compartment$.
\end{quote}
Note:
\begin{itemize}

\item this is the first type rule which has changed the state of
  the JS context from $S$ to $S'$,

\item although $S$ can access data, $S'$ cannot: this is necessary for
  safety, since the global does not yet have any data attached to it,
  so accessing it would be undefined behaviour,

\item $S'$ only has lifetime $\alpha$, so we do not have two JS contexts
  live simultaneously,

\item $S'$ has entered the compartment $C$, and has the permission to create
  new objects in $C$,
  
\item $S'$ has the permission $\IsInitializing{C,T}$, which allows the global
  to be initialized with native data of type $T$.
  
\end{itemize}
The type rule for initializing a compartment is:
\begin{quote}
  if $cx: \JSContext{S}$
  and $x: T$
  and $S: \IsInitializing{\alpha,C,T}$ \\
  then $cx.\globalManage(x) : \JSContext{S'}$ \\\mbox{}\quad
  where $S': \alpha + \CanAccess + \CanAlloc + \InCompartment{C}$
\end{quote}

\subsection{Entering a compartment}
\label{sec:compartment-entering}

Given a JS-managed reference \verb|x|, we can enter its compartment with \verb|cx.enter_known_compartment(x)|.
This returns a JS context whose current compartment is that of \verb|x|.
For example, given a JS-managed reference \verb|x|,
we can create a new JS-managed reference in the same compartment with:
\begin{verbatim}
  let ref mut cx = cx.enter_known_compartment(x);
  let y = cx.manage(String::from("hello"));
\end{verbatim}
This has type rule:
\begin{quote}
  if $cx: \REFMUT{\alpha}{\JSContext{S}}$
  and $x: \JSManaged{C,T}$ \\
  and $S: \CanAccess + \CanAlloc$
  and $C: \Compartment$ \\
  then $cx.\enterKnownCompartment(x) : \JSContext{S'}$ \\\mbox{}\quad
  where $S': \alpha + \CanAccess + \CanAlloc + \InCompartment{C}$
\end{quote}

\subsection{Wildcard compartments}
\label{sec:compartment-wildcard}

Working with named compartments is fine when there is a fixed number of
them, but not when the number of compartments is unbounded. For
example, the type \verb|Vec<JSManaged<C, T>>| contains a vector of managed
objects, all in the same compartment, but sometimes you need a vector
of objects in different compartments. This is the same problem that
existential polymorphism~\cite{expoly}, and in particular
Java wildcards~\cite[\S8.1.2]{jls} is designed to solve, and we adopt
a similar approach.

The wildcard is called \verb|SOMEWHERE|, which we will
often abbreviate as $\SOMEWHERE$.
$\JSManaged{\SOMEWHERE, T}$ refers to JS-managed data whose
compartment is unknown. For example \verb|Vec<JSManaged<?, T>>|
contains a vector of managed objects, which may all be in different
compartments.

To create a wildcard, we use $x.\forgetCompartment()$, with type
rule:
\begin{quote}
  if $x: \JSManaged{C,T}$ 
  then $x.\forgetCompartment(): \JSManaged{\SOMEWHERE,T[\SOMEWHERE/C]}$
\end{quote}
Entering a wildcard compartment is the same as for a known compartment,
but renames the compartment to a fresh name:
\begin{quote}
  if $cx: \REFMUT{\alpha}{\JSContext{S}}$
  and $x: \JSManaged{\SOMEWHERE,T}$ \\
  and $S: \CanAccess + \CanAlloc$ \\
  then $cx.\enterUnknownCompartment(x) : \JSContext{S'}$ \\\mbox{}\quad
  where $S': \alpha + \CanAccess + \CanAlloc + \InCompartment{D}$ \\\mbox{}\qquad
  for fresh $D: \Compartment$.
\end{quote}
We also have a function $cx.\entered()$ of type $\JSManaged{D,T[D/\SOMEWHERE]}$,
which gives access to $x$ in its newly named compartment.
Note that access to data in a wildcard compartment is not allowed
(in the type system this is enforced since we do not have $\SOMEWHERE:\Compartment$),
for example:
\begin{verbatim}
  fn example<S>(cx: &mut JSContext<S>, x: JSManaged<SOMEWHERE, String>) where
    S: CanAccess,
  {
    // We can't access x without first entering its compartment.
    // Commenting out the next two lines gives an error
    // the trait `Compartment` is not implemented for `SOMEWHERE`.
    let ref mut cx = cx.enter_unknown_compartment(x);
    let x = cx.entered();
    println!("Hello, {}.", x.borrow(cx));
  }
\end{verbatim}
