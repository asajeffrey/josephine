% \documentclass[acmsmall,review,anonymous]{acmart}
\documentclass[acmsmall]{acmart}
\settopmatter{printfolios=true,printacmref=false}

\usepackage{macros}

%% \acmJournal{PACMPL}
%% \acmVolume{1}
%% \acmNumber{CONF} % CONF = POPL or ICFP or OOPSLA
%% \acmArticle{1}
%% \acmYear{2018}
%% \acmMonth{1}
%% \acmDOI{} % \acmDOI{10.1145/nnnnnnn.nnnnnnn}
%% \startPage{1}

\setcopyright{none}
%\setcopyright{rightsretained}

\begin{document}

\title[Josephine]{Experience Report: Josephine}
\subtitle{Using JavaScript to safely manage the lifetimes of Rust data}

\author{Alan Jeffrey}
\orcid{0000-0001-6342-0318}
\affiliation{
  \position{Research Engineer}
  \institution{Mozilla Research}
}
\email{ajeffrey@mozilla.com}

\begin{abstract}
This paper is about the interface between languages
which use a garbage collector and those which use fancy
types for safe manual memory management.
It uses existing techniques for using linear capabilities
to provide safe access to copyable references,
but the application to languages
with a tracing garbage collector is new.
This work is in the area of mixed linear/non-linear
languages, but the linear language is Rust, and the
non-linear language is JavaScript.
\end{abstract}

\begin{CCSXML}
<ccs2012>
<concept>
<concept_id>10011007.10011006.10011008.10011009.10011012</concept_id>
<concept_desc>Software and its engineering~Functional languages</concept_desc>
<concept_significance>500</concept_significance>
</concept>
<concept>
<concept_id>10011007.10011006.10011008.10011009.10011010</concept_id>
<concept_desc>Software and its engineering~Imperative languages</concept_desc>
<concept_significance>300</concept_significance>
</concept>
</ccs2012>
\end{CCSXML}

\ccsdesc[500]{Software and its engineering~Functional languages}
\ccsdesc[300]{Software and its engineering~Imperative languages}

\keywords{JavaScript, Rust, interoperability, memory safety, affine types}

\maketitle

\section{Introduction}

This paper discusses the Josephine API~\cite{josephine} for using
the Spidermonkey~\cite{spidermonkey} JavaScript runtime to safely
manage the lifetime of Rust~\cite{rust} data. That sentence needs
some unpacking.

\subsection{Rust}

Rust is a systems programming language which uses fancy types to
ensure memory safety even in the presence of mutable update, and the
absence of a garbage collector. Rust has an affine type system, which
allows data to be discarded but does not allow data to be arbitrarily
copied. For example, the Rust program:
\begin{verbatim}
  let hello = String::from("hello");
  let moved = hello;
  println!("Oh look {} is hello", moved);
\end{verbatim}
is fine, but the program:
\begin{verbatim}
  let hello = String::from("hello");
  let copied = hello;
  println!("Oh look {} is {}", hello, copied);
\end{verbatim}
is not, since \verb|hello| and \verb|copied| are simultaneously live. Trying to compile
this program produces an error:
\begin{verbatim}
use of moved value: `hello`
 --> src/main.rs:4:32
  |
3 |   let copied = hello;
  |       ------ value moved here
4 |   println!("Oh look {} is {}", hello, copied);
  |                                ^^^^^ value used here after move
\end{verbatim}
The use of affine types allows aliasing to be tracked. For example, a
classic problem with aliasing is appending a string to itself. In
Rust, an example of appending a string is:
\begin{verbatim}
  let mut hello = String::from("hello");
  let ref world = String::from("world");
  hello.push_str(world);
  println!("Oh look hello is {}", hello);
\end{verbatim}
The important operation is \verb|hello.push_str(world)|, which mutates the
string \verb|hello| (hence the \verb|mut| annotation on the declaration of \verb|hello|).
The appended string \verb|world| is passed by reference,
(hence the \verb|ref| annotation on the declaration of \verb|world|).

A problem with mutably appending strings is ensuring that the string
is not appended to itself, for example the documentation for
C \verb|strcat|~\cite{strcat} states ``Source and destination may not
overlap'', but C does not check aliasing and relies on the programmer
to ensure correctness. In contrast, attempting to append a string
to itself in Rust:
\begin{verbatim}
  let ref mut hello = String::from("hello");
  hello.push_str(hello);
\end{verbatim}
produces an error:
\begin{verbatim}
cannot borrow `*hello` as immutable because it is also borrowed as mutable
 --> src/main.rs:3:18
  |
3 |   hello.push_str(hello);
  |   -----          ^^^^^- mutable borrow ends here
  |   |              |
  |   |              immutable borrow occurs here
  |   mutable borrow occurs here
\end{verbatim}
In Rust, the crucial invariant maintained by affine types is:
\begin{quote}\em
  Any memory that can be reached simultaneously by two different paths
  is immutable.
\end{quote}
For example in \verb|hello.push(hello)| there are two occurrences of \verb|hello| that
are live simultaneously, the first of which is mutating the string, so this is outlawed.

In order to track statically which variables are live simultaneously, Rust uses a lifetime
system similar to that used by region-based memory~\cite{regions}. Each allocation of
memory has a lifetime $\alpha$, and lifetimes are ordered $\alpha\subseteq\beta$.
Each code block introduces a lifetime, and for data which does not escape from its scope,
the nesting of blocks determines the ordering of lifetimes.

For example in the program:
\begin{verbatim}
  let ref x = String::from("hi");
  {
    let ref y = x;
    println!("y is {}", y);
  }
  println!("x is {}", x);
\end{verbatim}
the variable \verb|x| has a lifetime $\alpha$ given by the outer block,
and the variable \verb|y| has a lifetime $\beta\subseteq\alpha$ given by the inner block.

These lifetimes are mentioned in the types of references: the type $\REF\alpha T$
is a reference giving immutable access to data of type $T$, which will live at least as long as
$\alpha$. Similarly, the type $\REFMUT\alpha T$ gives mutable access to the data: the crucial
difference is that $\REF\alpha T$ is a copyable type, but $\REFMUT\alpha T$ is not.
For example
the type of \verb|x| is $\REF\alpha\STRING$ and the type of \verb|y| is
$\REF\beta(\REF\alpha\STRING)$, which is well-formed because $\beta\subseteq\alpha$.

Lifetimes are used to prevent another class of memory safety issues: use-after-free.
For example, consider the program:
\begin{verbatim}
  let hi = String::from("hi");
  let ref mut handle = &hi;
  {
    let lo = String::from("lo");
    *handle = &lo;
  } // lo is deallocated here
  println!("handle is {}", **handle);
\end{verbatim}
If this program were to execute, its behaviour would be undefined,
since \verb|**handle| is used after \verb|lo|
(which \verb|handle| points to) is deallocated. Fortunately, this program
does not pass Rust's borrow-checker:
\begin{verbatim}
`lo` does not live long enough
 --> src/main.rs:6:11
  |
6 |     *handle = &lo;
  |                ^^ borrowed value does not live long enough
7 |   } // lo is deallocated here
  |   - `lo` dropped here while still borrowed
8 |   println!("handle is {}", **handle);
9 | }
  | - borrowed value needs to live until here
\end{verbatim}
This use-after-free error can be detected because (naming the outer lifetime to be
$\alpha$ and the inner lifetime to be $\beta\subseteq\alpha$) \verb|handle| has type
$\REFMUT\alpha\REF\alpha\STRING$, but \verb|&lo| only has type $\REF\beta\STRING$, no
$\REF\alpha\STRING$ as required by the assignment.

Lifetimes avoid use-after-free by maintaining two invariants:
\begin{quote}\em
  Any dereference happens during the lifetime of the reference, \\
  and deallocation happens after the lifetime of all references.
\end{quote}
There is more to the Rust type system than described here
(higher-order functions, traits, variance, \dots) but the important features
are \emph{affine types} and \emph{lifetimes} for ensuring memory safety,
even in the presence of manual memory management.

\section{The Josephine API}

There are two important concepts in Josephine's API: \emph{JS-managed} data,
and the JS \emph{context}. For readers familiar with the region-based
variant~\cite{l3-with-regions} of $L^3$~\cite{l3}, JS-managed data
corresponds to $L^3$ references, and JS contexts to $L^3$ capabilities.

\subsection{JS-managed data}

JS-managed data has the type $\JSManaged{\alpha, C, T}$, which represents
a reference to data whose lifetime is managed by JS, which:
\begin{itemize}

\item is guaranteed to live at least as long as $\alpha$,
\item is allocated in JS compartment $C$,
\item points to native data of type $T$.
  
\end{itemize}
This type is copyable, so not subject to the affine type discipline,
even though it can be used to gain mutable access to the native
data. We shall see later that this is safe for the same reason as
$L^3$: we are using the JS context as a capability, and it is not
copyable.

In examples, we make use of Rust's \emph{lifetime elision}~\cite{???},
and just write $\JSManaged{C,T}$ where the lifetime $\alpha$ can be
inferred.

In the simplest case, $T$ is a base type like $\STRING$, but in more complex
cases, $T$ might itself contain JS-managed data, for example a type of
cells in a doubly-linked list can be defined:
\begin{verbatim}
  type Cell<'a, C> = JSManaged<'a, C, NativeCell<'a, C>>;
\end{verbatim}
where:
\begin{verbatim}
  struct NativeCell<'a, C> {
    data: String,
    prev: Option<Cell<'a, C>>,
    next: Option<Cell<'a, C>>,
  }
\end{verbatim}
This pattern is a common idiom, in that there are two types:
\begin{itemize}
\item $\NativeCell{\alpha,C}$ containing the native representation
of a cell, including the prev and next
references, and
\item $\Cell{\alpha,C}$ containing a reference to a native cell,
whose lifetime is managed by JS.
\end{itemize}
These types are both parameterized by a lower bound $\alpha$ on the lifetime
of the cell, and the compartment $C$ that the cell lives in.

Doubly-linked lists are an interesting example of programming in Rust,
and indeed there is an introductory text \emph{Learning Rust With
  Entirely Too Many Linked Lists}~\cite{too-many-lists}, in which safe
implementations of doubly-linked lists require interior mutability
(and hence dynamic checks) and reference counting.

\subsection{The JS context}

By itself, JS-managed references are not much use: there has to be an
API for creating and dereferencing them: this is the role of the
JS \emph{context}, which acts as a capability for manipulating
JS-managed data.

There is only one JS context per thread (and JS contexts cannot be shared
or sent between threads) so unique access to the JS context implies unique
access to all JS-managed data. We can use this to give safe mutable access
to JS-managed data, since the JS context is a unique capability.

The JS context has a state, notably keeping track of the current
compartment, but also permissions such as ``allowed to create new
references'' or ``allowed to dereference''.  This state is tracked in
the type system using phantom types~\cite{phantom}, so the JS context
has type $\JSContext{S}$, where $S$ is the current state.

For example, a program to allocate a new JS-managed reference is:
\begin{verbatim}
  let x: JSManaged<C, String> = cx.manage(String::from("hello"));
\end{verbatim}
and a program to access a JS-managed reference is:
\begin{verbatim}
  let msg: &String = x.borrow(cx);
\end{verbatim}
These programs make use of the JS context \verb|cx|. In order for the
first example to typecheck:
\begin{itemize}

\item \verb|cx| must have type $\REFMUT{}\JSContext{S}$, where
\item $S$ (the state of the context) must have permission to allocate
  references in $C$, and
\item $C$ must be a compartment.

\end{itemize}
The second example is similar, except:
\begin{itemize}

\item we do not need mutable access to the context, and
\item $S$ need the permission to access compartment $C$.

\end{itemize}
Fortunately, Rust has a \emph{trait} system (similar to Haskell's
class system), which allows us to express these constraints.  In the
same way that $C$ and $S$ are phantom types, these are \emph{marker}
traits with no computational value. The typing for
the first example is:
\begin{verbatim}
  (cx: &mut JSContext<S>) where
    S: CanAlloc + InCompartment<C>,
    C: Compartment,
\end{verbatim}
and for the second:
\begin{verbatim}
  (cx: &JSContext<S>) where
    S: CanAccess,
    C: Compartment,
\end{verbatim}
A program to mutably access a JS-managed reference is:
\begin{verbatim}
  let msg: &mut String = x.borrow_mut(cx);
\end{verbatim}
at which point the fact that the JS context is an affine capability
becomes important. The typing required for this is:
\begin{verbatim}
  (cx: &mut JSContext<S>) where
    S: CanAccess,
    C: Compartment,
\end{verbatim}
That is \emph{unique access to JS-managed data requires unique access to th4e JS context},
and so we cannot simultaneously have mutable access to two different JS-managed
references. This is the same safety condition that region-based $L^3$ uses.

For example, we can use this (together with Rust's built-in replace function
which swaps the contents of a mutable reference) to replace the contents of a cell
with a new value:
\begin{verbatim}
  fn replace<S>(self, cx: &'a mut JSContext<S>, new_data: String) -> String where
    S: CanAccess,
    C: Compartment,
  {
    let ref mut old_data = self.0.borrow_mut(cx).data;
    replace(old_data, new_data)
  }
\end{verbatim}

\section{Interfacing to the garbage collector}

Interfacing to the Spidermonkey GC has two parts:
\begin{itemize}

\item \emph{tracing}: from a JS-managed reference, find
  the JS-managed references immediately reachable from it, and

\item \emph{rooting}: find the JS-managed referenes which
  are reachable from the stack.

\end{itemize}
From these two functions, it is possible to find all of the
JS-managed references which are reachable from Rust. Together
with Spidermonkey's regular GC, this allows the runtime to
find all of the reachable JS objects, and then to reclaim the
unreachable ones.

These interfaces are important for safety, since
under-approximation can result in use-after-free,
and over-approximation can result in space leaks.

In this section, we discuss how Josephine supports these interfaces.

\subsection{Tracing}

Interfacing to the Spidermonkey tracer via Josephine is achieved
by implementing a trait:
\begin{verbatim}
  pub unsafe trait JSTraceable {
    unsafe fn trace(&self, trc: *mut JSTracer);
  }
\end{verbatim}
Josephine provides an implementation:
\begin{verbatim}
  unsafe impl<'a, C, T> JSTraceable for JSManaged<'a, C, T> where
    T: JSTraceable { ... }
\end{verbatim}
User-defined types can then implement the interface by
recursively visiting fields, for example:
\begin{verbatim}
  unsafe impl<'a, C, T> JSTraceable for NativeCell<'a, C> {
    unsafe fn trace(&self, trc: *mut JSTracer) {
      self.prev.trace(trc);
      self.next.trace(trc);
    }
  }
\end{verbatim}
This is a lot of unsafe boilerplate, but fortunately can
also be mechanized using meta-programming by marking a type
as \verb|#[derive(JSTraceable)]|.

One subtlety is that during tracing data of type $T$, the JS runtime
has a reference of type $\REF{}{T}$ given by the \verb|self| parameter
to \verb|trace|. For this to be safe, we have to ensure that there is
no mutable reference to that data. This is maintained by the
previously mentioned invariant:
\begin{quote}\em
  Any operation that can trigger garbage collection
  requires mutable access to the JS context.
\end{quote}
Tracing is triggered by garbage collection, and so had unique access
to the JS context, so there cannot be any other live mutable access
to any JS-managed data.

\subsection{Rooting}

In languages with native support for GC, rooting is
supported by the compiler, which can provide metadata for
each stack frame allowing it to be traced. In languages like
Rust that do not have a native GC, this metadata is not
present, and instead rooting has to be performed explicitly.

This explicit rooting is needed whenever an object is
needed to outlive the borrow of the JS context that produced
it. For example, a function to insert a new cell after
an existing one is:
\begin{verbatim}
  pub fn insert<C, S>(cell: Cell<C>, data: String, cx: &mut JSContext<S>) where
    S: CanAccess + CanAlloc + InCompartment<C>,
    C: Compartment,
  {
    let old_next = cell.borrow(cx).next;
    let new_next = cx.manage(NativeCell {
      data: data,
      prev: Some(cell),
      next: old_next,
    });
    cell.borrow_mut(cx).next = Some(new_next);
    if let Some(old_next) = old_next {
      old_next.borrow_mut(cx).prev = Some(new_next);
    }
  }
\end{verbatim}
This code is the ``code you would first think of'' for inserting an
element into a doubly-linked list, but is in fact not safe because
the local variables \verb|old_next| and \verb|new_next| have not been
rooted. If GC were triggered just after \verb|new_next| was created,
then it could be reclaimed, causing a later use-after-free.

Fortunately, Josephine will catch these safety problems, and report
errors such as:
\begin{verbatim}
  error[E0502]: cannot borrow `*cx` as mutable because
    it is also borrowed as immutable
    |
    |         let old_next = self.borrow(cx).next;
    |                                    -- immutable borrow occurs here
    |         let new_next = cx.manage(NativeCell {
    |                        ^^ mutable borrow occurs here
...
    |     }
    |     - immutable borrow ends here
\end{verbatim}
The fix is to explictly root the local variables. In Josephine this is:
\begin{verbatim}
   let ref mut root1 = cx.new_root();
   let ref mut root2 = cx.new_root();
   let old_next = (... as before ...).in_root(root1);
   let new_next = (... as before ...).in_root(root2);
\end{verbatim}
The declaration of a \verb|root| allocates space on the stack
for a new root, and \verb|managed.in_root(root)| roots \verb|managed|.
Note that it is just the reference that is copied to the stack,
the JS-managed data itself doesn't move.
Roots have type $\JSRoot{\beta,T}$ where $\beta$ is the lifetime
of the root, and $T$ is the type being rooted.

Once the local variables are rooted, the code typecheck,
because rooting changes the lifetime of the JS-managed
data, for example:
\begin{quote}
  if~$p: \JSManaged{\alpha,C,T}$ \\
  and~$r: \REFMUT{\beta}{\JSRoot{\beta,\JSManaged{\beta,C,T[\beta/\alpha]}}}$ \\
  then~$p.\inRoot(r): \JSManaged{\beta,C,T[\beta/\alpha]}$.
\end{quote}
Before rooting, the JS-managed data had lifetime $\alpha$,
which is usually the lifetime of the borrow of the JS context
that created or accessed it.
After rooting, the JS-managed data has lifetime $\beta$,
which is the lifetime of the root itself. Since roots are
considered reachable by GC, the contents of a root
are guaranteed not to be GC'd during its lifetime,
so this rule is sound.

\section{Conclusions}

The contributions of this work are:
\begin{itemize}

\item An implementation of the ideas in $L^3$~\cite{l3} to mixed
  linear/non-linear programming~\cite{mixed}, where the
  linear language is Rust and the non-linear language is
  JavaScript.

\item A treatment of garbage collection (rather than region-based
  memory management) for such a system.

\item A treatment of how operations such as accessing, mutating, and
  rooting can change the lifetimes of objects.
  
\end{itemize}
The main item left for future work is formalizing the approach
described here: memory safety is conjectured, but not proved
formally.

There are some aspects of the API which need more investigation:
\begin{itemize}

\item Other JavaScript engines take a different approach to
  rooting, notably V8 \emph{handle scopes}~\cite{v8-embedding},
  which have different trade-offs. In terms of this paper, the
  roots are attached to the JS context, rather than stored
  on the stack. It would be interesting to compare these approaches.

\item Josephine uses phantom types to track which compartment
  memory is allocated in, but does not support features such
  as \emph{cross-compartment wrappers}~\cite{compartments},
  which allow references between compartments.

\item In this paper, we have just used the SpiderMonkey runtime
  engine for its garbage collector, but it is a full-featured
  JavaScript engine, and it would be good to provide safe
  access to executing JS code. This would be simpler to achieve
  if there were a JS type system to generate bindings from,
  such as TypeScript~\cite{typescript}.
  
\end{itemize}
The distribution includes some simple examples such as doubly-linked lists
and a stripped-down DOM, but more examples are needed to see if the API
is usable for practical code.


\bibliographystyle{plain}
\bibliography{paper.bib}

\end{document}
