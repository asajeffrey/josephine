% \documentclass[acmsmall,review,anonymous]{acmart}
\documentclass[acmsmall]{acmart}
\settopmatter{printfolios=true,printacmref=false}

\usepackage{macros}

%% \acmJournal{PACMPL}
%% \acmVolume{1}
%% \acmNumber{CONF} % CONF = POPL or ICFP or OOPSLA
%% \acmArticle{1}
%% \acmYear{2018}
%% \acmMonth{1}
%% \acmDOI{} % \acmDOI{10.1145/nnnnnnn.nnnnnnn}
%% \startPage{1}

%\setcopyright{none}
\setcopyright{rightsretained}

\begin{document}

\title{Experience Report: Josephine}
\subtitle{Using JavaScript to safely manage the lifetimes of Rust data}

\author{Alan Jeffrey}
\orcid{0000-0001-6342-0318}
\affiliation{
  \position{Research Engineer}
  \institution{Mozilla Research}
}
\email{ajeffrey@mozilla.com}

\begin{abstract}
This paper is about the interface between languages
which use a garbage collector and those which use fancy
types for safe manual memory management.
It uses existing techniques for using linear capabilities
to provide safe access to copyable references,
but the application to languages
with a tracing garbage collector,
and to data with explicit lifetimes is new.
This work is in the area of mixed linear/non-linear
languages, but the linear language is Rust, and the
non-linear language is JavaScript.
\end{abstract}

\begin{CCSXML}
<ccs2012>
<concept>
<concept_id>10011007.10011006.10011008.10011009.10011012</concept_id>
<concept_desc>Software and its engineering~Functional languages</concept_desc>
<concept_significance>500</concept_significance>
</concept>
<concept>
<concept_id>10011007.10011006.10011008.10011009.10011010</concept_id>
<concept_desc>Software and its engineering~Imperative languages</concept_desc>
<concept_significance>300</concept_significance>
</concept>
</ccs2012>
\end{CCSXML}

\ccsdesc[500]{Software and its engineering~Functional languages}
\ccsdesc[300]{Software and its engineering~Imperative languages}

\keywords{JavaScript, Rust, interoperability, memory management, affine types}

\maketitle

\section{Introduction}

This paper is about the interface between languages
which use a garbage collector and those which use fancy
types for safe manual memory management.

Garbage collection is the most common memory management technique for
functional programming languages, dating back to LISP~\cite{LISP}.
Having a garbage collector guarantees memory safety, but at the
cost of having a required runtime system.

Imperative languages often require the programmer to perform
manual memory management, such as the \verb|malloc| and \verb|free|
functions provided by C~\cite{K+R}. The safety of a program
(in particular the absence of \emph{use-after-free} errors)
is considered the programmer's problem.
More recently, laguages such as Cyclone~\cite{cyclone}
and Rust~\cite{rust} have used fancy type systems
such as substructural types~\cite{girard,Go4,walker}
and region analysis~\cite{regions} to guarantee memory
safety without garbage collection.

This paper discusses the Josephine API~\cite{josephine} for using the
garbage collector provided by the Spidermonkey~\cite{spidermonkey}
JavaScript runtime to safely manage the lifetime of Rust~\cite{rust}
data. It uses techniques from $L^3$~\cite{l3} and its application
to regions~\cite{l3-with-regions}, but the application to languages
with a tracing garbage collector is new.

\subsection{Rust}

Rust is a systems programming language which uses fancy types to
ensure memory safety even in the presence of mutable update, and
manual memory management. Rust has an affine type system, which
allows data to be discarded but does not allow data to be arbitrarily
copied. For example, the Rust program:
\begin{verbatim}
  let hello = String::from("hello");
  let moved = hello;
  println!("Oh look {} is hello", moved);
\end{verbatim}
is fine, but the program:
\begin{verbatim}
  let hello = String::from("hello");
  let copied = hello;
  println!("Oh look {} is {}", hello, copied);
\end{verbatim}
is not, since \verb|hello| and \verb|copied| are simultaneously live. Trying to compile
this program produces:
\begin{verbatim}
  use of moved value: `hello`
   --> src/main.rs:4:32
    |
  3 |   let copied = hello;
    |       ------ value moved here
  4 |   println!("Oh look {} is {}", hello, copied);
    |                                ^^^^^ value used here after move
\end{verbatim}
The use of affine types allows aliasing to be tracked. For example, a
classic problem with aliasing is appending a string to itself. In
Rust, an example of appending a string is:
\begin{verbatim}
  let mut hello = String::from("hello");
  let ref world = String::from("world");
  hello.push_str(world);
  println!("Oh look hello is {}", hello);
\end{verbatim}
The important operation is \verb|hello.push_str(world)|, which mutates the
string \verb|hello| (hence the \verb|mut| annotation on the declaration of \verb|hello|).
The appended string \verb|world| is passed by reference,
(hence the \verb|ref| annotation on the declaration of \verb|world|).

A problem with mutably appending strings is ensuring that the string
is not appended to itself, for example the documentation for
C \verb|strcat| states ``Source and destination may not
overlap'', but C does not check aliasing and relies on the programmer
to ensure correctness. In contrast, attempting to append a string
to itself in Rust:
\begin{verbatim}
  let ref mut hello = String::from("hello");
  hello.push_str(hello);
\end{verbatim}
produces an error:
\begin{verbatim}
  cannot borrow `*hello` as immutable because it is also borrowed as mutable
   --> src/main.rs:3:18
    |
  3 |   hello.push_str(hello);
    |   -----          ^^^^^- mutable borrow ends here
    |   |              |
    |   |              immutable borrow occurs here
    |   mutable borrow occurs here
\end{verbatim}
In Rust, the crucial invariant maintained by affine types is:
\begin{quote}\em
  Any memory that can be reached simultaneously by two different paths
  is immutable.
\end{quote}
For example in \verb|hello.push(hello)| there are two occurrences of \verb|hello| that
are live simultaneously, the first of which is mutating the string, so this is outlawed.

In order to track statically which variables are live simultaneously, Rust uses a lifetime
system similar to that used by region-based memory~\cite{regions}. Each allocation of
memory has a lifetime $\alpha$, and lifetimes are ordered $\alpha\subseteq\beta$.
Each code block introduces a lifetime, and for data which does not escape from its scope,
the nesting of blocks determines the ordering of lifetimes.

For example in the program:
\begin{verbatim}
  let ref x = String::from("hi");
  {
    let ref y = x;
    println!("y is {}", y);
  }
  println!("x is {}", x);
\end{verbatim}
the variable \verb|x| has a lifetime $\alpha$ given by the outer block,
and the variable \verb|y| has a lifetime $\beta\subseteq\alpha$ given by the inner block.

These lifetimes are mentioned in the types of references: the type $\REF\alpha T$
is a reference giving immutable access to data of type $T$, which will live at least as long as
$\alpha$. Similarly, the type $\REFMUT\alpha T$ gives mutable access to the data: the crucial
difference is that $\REF\alpha T$ is a copyable type, but $\REFMUT\alpha T$ is not.
For example
the type of \verb|x| is $\REF\alpha\STRING$ and the type of \verb|y| is
$\REF\beta(\REF\alpha\STRING)$, which is well-formed because $\beta\subseteq\alpha$.

Lifetimes are used to prevent another class of memory safety issues: use-after-free.
For example, consider the program:
\begin{verbatim}
  let hi = String::from("hi");
  let ref mut handle = &hi;
  {
    let lo = String::from("lo");
    *handle = &lo;
  } // lo is deallocated here
  println!("handle is {}", **handle);
\end{verbatim}
If this program were to execute, its behaviour would be undefined,
since \verb|**handle| is used after \verb|lo|
(which \verb|handle| points to) is deallocated. Fortunately, this program
does not pass Rust's borrow-checker:
\begin{verbatim}
  `lo` does not live long enough
   --> src/main.rs:6:11
    |
  6 |     *handle = &lo;
    |                ^^ borrowed value does not live long enough
  7 |   } // lo is deallocated here
    |   - `lo` dropped here while still borrowed
  8 |   println!("handle is {}", **handle);
  9 | }
    | - borrowed value needs to live until here
\end{verbatim}
This use-after-free error can be detected because (naming the outer lifetime to be
$\alpha$ and the inner lifetime to be $\beta\subseteq\alpha$) \verb|handle| has type
$\REFMUT\alpha\REF\alpha\STRING$, but \verb|&lo| only has type $\REF\beta\STRING$, no
$\REF\alpha\STRING$ as required by the assignment.

Lifetimes avoid use-after-free by maintaining two invariants:
\begin{quote}\em
  Any dereference happens during the lifetime of the reference, \\
  and deallocation happens after the lifetime of all references.
\end{quote}
There is more to the Rust type system than described here
(higher-order functions, traits, variance, concurrency, \dots) but the important features
are \emph{affine types} and \emph{lifetimes} for ensuring memory safety,
even in the presence of manual memory management.

\subsection{Spidermonkey}

Spidermonkey is Mozilla's JavaScript runtime, used in the Firefox browser,
and the Servo~\cite{servo} next-generation web engine. This is a full-featured
JS implementation, but the focus of this paper is its automatic memory management.

Inside a web engine, there are often native implementations of HTML features,
which are exposed to JavaScript using DOM interfaces. For example, an HTML image
is exposed to JavaScript as a DOM object representing an \verb|<img>| element,
but behind the scenes there is native code responsible for loading and rendering
images.

When a JavaScript object is garbage collected, a destructor is called to
deallocate any attached native memory. In the case that the native code
is implemented in Rust, this leads to a situation where Rust relies on affine
types and lifetimes for memory safety, but JavaScript respects neither of these.
As a result, the raw Spidermonkey interface to Rust is very unsafe,
for example there are nearly 400 instances of unsafe code in the Servo
DOM implementation:
\begin{verbatim}
  $ grep "unsafe_code" components/script/dom/*.rs | wc
      393     734   25514
\end{verbatim}
Since JavaScript does not respect Rust's affine types,
Servo's DOM implementation makes use of Rust~\cite[\S3.11]{rust}
\emph{interior mutability} which replaces the compile-time type
checks with run-time dynamic checks. This carries run-time overhead,
and the possibility of checks failing, and Servo panicking.

Moreover, Spidermonkey has its own invariants, and if an embedding
application does not respect these invariants, then runtime errors can
occur. One of these invariants is the division of JavaScript memory
into \emph{compartments}, which can be garbage collected
separately. The runtime has a notion of ``current compartment'',
and the embedding application is asked to maintain two invariants:
\begin{itemize}
  \item whenever an object is used, the object is in the current compartment, and
  \item there are no references between objects which cross compartments.
\end{itemize}
In order for native code to interact well with the Spidermonkey garbage collector,
it has to provide two functions:
\begin{itemize}
\item a \emph{trace} function, that given an object, iterates over all of the
  JavaScript objects which are reachable from it, and
\item a \emph{roots} function, which iterates over all of the JavaScript
  objects that are live on the call stack.
\end{itemize}
From these two functions, the garbage collector can find all of the reachable
JavaScript objects, including those reachable from JavaScript directly, and
those reached via native code.

Automatically generating the trace function is reasonably straightforward
metaprogramming, but rooting safely turns out to be quite tricky.
Servo provides an approximate analysis for safe rooting using an ad-hoc
static analysis (the \emph{rooting lint}), but this is problematic because
a) the lint is syntax-driven, so does not understand about Rust features
such as generics, and b) even if it could be made sound it is disabled
more than 200 times:
\begin{verbatim}
  $ grep "unrooted_must_root" components/script/dom/*.rs | wc
      213     456   15961
\end{verbatim}

\subsection{Josephine}

Josephine~\cite{josephine} is intended to act as a safe bridge between
Spidermonkey and Rust. Its goals are:
\begin{itemize}

\item to use JavaScript to manage the lifetime of Rust data,
  and to allow JavaScript to garbage collect unreachable data,

\item to allow references to JavaScript-managed data to be freely copied and discarded,
  relying on Spidermonkey's garbage collector for safety,

\item to maintain Rust memory safety via affine types and lifetimes,
  without requiring additional static analysis such as the rooting lint,

\item to allow mutable and immutable access to Rust data via JavaScript-managed references,
  so avoiding interior mutability, and

\item to provide a rooting API to ensure that JavaScript-managed data is not garbage collected
  while it is being used.

\end{itemize}
Josephine is intended to be safe, in that any programs built using Josephine's API
do not introduce undefined behaviour or runtime errors.
Josephine achieves this by providing controlled access to
Spidermonkey's \emph{JavaScript context}, and maintaining invariants about it:
\begin{itemize}

\item immutable access to JavaScript-managed data requires immutable access
  to the JavaScript context,

\item mutable access to JavaScript-managed data requires mutable access
  to the JavaScript context, and

\item any action that can trigger garbage collection (for example allocating
  new objects) requires mutable access to the JavaScript context.

\end{itemize}
As a result, since access to the JavaScript context does respect
Rust's affine types, mutation or garbage collection cannot occur
simultaneously with accessing JavaScript-managed data.

In other words, Josephine treats the JavaScript context as an affine
access token, or capability, which controls access to the JavaScript-managed
data. The data accesses respect affine types, even though the JavaScript objects
themselves do not.

This use of an access token to safely access data in a substructural
type system is \emph{not} new, it is the heart of Ahmed, Fluet and
Morrisett's $L^3$ Linear Language with Locations~\cite{l3} and its
application to linear regions~\cite{regions}.

Moreover, type systems for mixed linear/non-linear programming have
been known for more than 20 years~\cite{mixed}.
The aspects that are novel are:
\begin{itemize}

\item the linear language is Rust, and the non-linear language is
  JavaScript, which are both industrial-strength languages,

\item the treatment of garbage collection requires a
  different treatment than regions in $L^3$, which have a stack
  discipline, and

\item the token contains more state in its type, carrying more than
  just read/write access, but the current compartment, and the capability
  to trigger garbage collection.

\end{itemize}

\subsection*{Acknowledgments}

This work benefited greatly from conversations with
Amal Ahmed,
Nick Benton,
Josh Bowman-Matthews,
Manish Goregaokar,
Bobby Holly, and
Anthony Ramine.

\section{The Josephine API}

There are two important concepts in Josephine's API: \emph{JS-managed} data,
and the JS \emph{context}. For readers familiar with the region-based
variant~\cite{l3-with-regions} of $L^3$~\cite{l3}, JS-managed data
corresponds to $L^3$ references, and JS contexts to $L^3$ capabilities.

\subsection{JS-managed data}

JS-managed data has the type $\JSManaged{\alpha, C, T}$, which represents
a reference to data whose lifetime is managed by JS, which:
\begin{itemize}

\item is guaranteed to live at least as long as $\alpha$,
\item is allocated in JS compartment $C$,
\item points to native data of type $T$.
  
\end{itemize}
This type is copyable, so not subject to the affine type discipline,
even though it can be used to gain mutable access to the native
data. We shall see later that this is safe for the same reason as
$L^3$: we are using the JS context as a capability, and it is not
copyable.

In the simplest case, $T$ is a base type like $\STRING$, but in more complex
cases, $T$ might itself contain JS-managed data, for example a type of
cells in a doubly-linked list can be defined:
\begin{verbatim}
  type Cell<'a, C> = JSManaged<'a, C, NativeCell<'a, C>>;
\end{verbatim}
where:
\begin{verbatim}
  struct NativeCell<'a, C> {
    data: String,
    prev: Option<Cell<'a, C>>,
    next: Option<Cell<'a, C>>,
  }
\end{verbatim}
This pattern is a common idiom, in that there are two types:
\begin{itemize}
\item $\NativeCell{\alpha,C}$ containing the native representation
of a cell, including the prev and next
references, and
\item $\Cell{\alpha,C}$ containing a reference to a native cell,
whose lifetime is managed by JS.
\end{itemize}
These types are both parameterized by a lower bound $\alpha$ on the lifetime
of the cell, and the compartment $C$ that the cell lives in.

Doubly-linked lists are an interesting example of programming in Rust,
and indeed there is an introductory text \emph{Learning Rust With
  Entirely Too Many Linked Lists}~\cite{too-many-lists}, in which safe
implementations of doubly-linked lists require interior mutability
(and hence dynamic checks) and reference counting.

\subsection{The JS context}

By itself, JS-managed references are not much use: there has to be an
API for creating and dereferencing them: this is the role of the
JS \emph{context}, which acts as a capability for manipulating
JS-managed data.

There is only one JS context per thread (and JS contexts cannot be shared
or sent between threads) so unique access to the JS context implies unique
access to all JS-managed data. We can use this to give safe mutable access
to JS-managed data, since the JS context is a unique capability.

The JS context has a state, notably keeping track of the current
compartment, but also permissions such as ``allowed to create new
references'' or ``allowed to dereference''.  This state is tracked in
the type system using phantom types~\cite{phantom}, so the JS context
has type $\JSContext{S}$, where $S$ is the current state.

For example, a program to allocate a new JS-managed reference is:
\begin{verbatim}
  let x: JSManaged<C, String> = cx.manage(String::from("hello"));
\end{verbatim}

\section{Interfacing to the garbage collector}

Interfacing to the SpiderMonkey GC has two parts:
\begin{itemize}

\item \emph{tracing}: from a JS-managed reference, find
  the JS-managed references immediately reachable from it, and

\item \emph{rooting}: find the JS-managed references which
  are reachable from the stack.

\end{itemize}
From these two functions, it is possible to find all of the
JS-managed references which are reachable from Rust. Together
with SpiderMonkey's regular GC, this allows the runtime to
find all of the reachable JS objects, and then to reclaim the
unreachable ones.

These interfaces are important for safety, since
under-approximation can result in use-after-free,
and over-approximation can result in space leaks.

In this section, we discuss how Josephine supports these interfaces.

\subsection{Tracing}

Interfacing to the SpiderMonkey tracer via Josephine is achieved
by implementing a trait:
\begin{verbatim}
  pub unsafe trait JSTraceable {
    unsafe fn trace(&self, trc: *mut JSTracer);
  }
\end{verbatim}
Josephine provides an implementation:
\begin{verbatim}
  unsafe impl<'a, C, T> JSTraceable for JSManaged<'a, C, T> where
    T: JSTraceable { ... }
\end{verbatim}
User-defined types can then implement the interface by
recursively visiting fields, for example:
\begin{verbatim}
  unsafe impl<'a, C, T> JSTraceable for NativeCell<'a, C> {
    unsafe fn trace(&self, trc: *mut JSTracer) {
      self.prev.trace(trc);
      self.next.trace(trc);
    }
  }
\end{verbatim}
This is a lot of unsafe boilerplate, but fortunately can
also be mechanized using meta-programming by marking a type
as \verb|#[derive(JSTraceable)]|.

One subtlety is that during tracing data of type $T$, the JS runtime
has a reference of type $\REF{}{T}$ given by the \verb|self| parameter
to \verb|trace|. For this to be safe, we have to ensure that there is
no mutable reference to that data. This is maintained by the
previously mentioned invariant:
\begin{quote}\em
  Any operation that can trigger garbage collection
  requires mutable access to the JS context.
\end{quote}
Tracing is triggered by garbage collection, and so had unique access
to the JS context, so there cannot be any other live mutable access
to any JS-managed data.

\subsection{Rooting}

In languages with native support for GC, rooting is
supported by the compiler, which can provide metadata for
each stack frame allowing it to be traced. In languages like
Rust that do not have a native GC, this metadata is not
present, and instead rooting has to be performed explicitly.

This explicit rooting is needed whenever an object is
needed to outlive the borrow of the JS context that produced
it. For example, a function to insert a new cell after
an existing one is:
\begin{verbatim}
  pub fn insert<C, S>(cell: Cell<C>, data: String, cx: &mut JSContext<S>) where
    S: CanAccess + CanAlloc + InCompartment<C>,
    C: Compartment,
  {
    let old_next = cell.borrow(cx).next;
    let new_next = cx.manage(NativeCell {
      data: data,
      prev: Some(cell),
      next: old_next,
    });
    cell.borrow_mut(cx).next = Some(new_next);
    if let Some(old_next) = old_next {
      old_next.borrow_mut(cx).prev = Some(new_next);
    }
  }
\end{verbatim}
This code is the ``code you would first think of'' for inserting an
element into a doubly-linked list, but is in fact not safe because
the local variables \verb|old_next| and \verb|new_next| have not been
rooted. If GC were triggered just after \verb|new_next| was created,
then it could be reclaimed, causing a later use-after-free.

Fortunately, Josephine will catch these safety problems, and report
errors such as:
\begin{verbatim}
  error[E0502]: cannot borrow `*cx` as mutable because
    it is also borrowed as immutable
    |
    |         let old_next = self.borrow(cx).next;
    |                                    -- immutable borrow occurs here
    |         let new_next = cx.manage(NativeCell {
    |                        ^^ mutable borrow occurs here
...
    |     }
    |     - immutable borrow ends here
\end{verbatim}
The fix is to explicitly root the local variables. In Josephine this is:
\begin{verbatim}
   let ref mut root1 = cx.new_root();
   let ref mut root2 = cx.new_root();
   let old_next = (... as before ...).in_root(root1);
   let new_next = (... as before ...).in_root(root2);
\end{verbatim}
The declaration of a \verb|root| allocates space on the stack
for a new root, and \verb|managed.in_root(root)| roots \verb|managed|.
Note that it is just the reference that is copied to the stack,
the JS-managed data itself doesn't move.
Roots have type $\JSRoot{\beta,T}$ where $\beta$ is the lifetime
of the root, and $T$ is the type being rooted.

Once the local variables are rooted, the code typecheck,
because rooting changes the lifetime of the JS-managed
data, for example:
\begin{quote}
  if~$p: \JSManaged{\alpha,C,T}$ \\
  and~$r: \REFMUT{\beta}{\JSRoot{\beta,\JSManaged{\beta,C,T[\beta/\alpha]}}}$ \\
  then~$p.\inRoot(r): \JSManaged{\beta,C,T[\beta/\alpha]}$.
\end{quote}
Before rooting, the JS-managed data had lifetime $\alpha$,
which is usually the lifetime of the borrow of the JS context
that created or accessed it.
After rooting, the JS-managed data has lifetime $\beta$,
which is the lifetime of the root itself. Since roots are
considered reachable by GC, the contents of a root
are guaranteed not to be GC'd during its lifetime,
so this rule is sound.

\section{Conclusions}

The contributions of this work are:
\begin{itemize}

\item An implementation of the ideas in $L^3$~\cite{l3} to mixed
  linear/non-linear programming~\cite{mixed}, where the
  linear language is Rust and the non-linear language is
  JavaScript.

\item A treatment of garbage collection (rather than region-based
  memory management) for such a system.

\item A treatment of how operations such as accessing, mutating, and
  rooting can change the lifetimes of objects.
  
\end{itemize}

Formalization.

Rooting scopes.

Cross-compartment wrappers.

FFI.

Scale up.


\bibliographystyle{plain}
\bibliography{paper.bib}

\end{document}
