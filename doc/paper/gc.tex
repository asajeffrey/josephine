\section{Interfacing to the garbage collector}

Interfacing to the SpiderMonkey GC has two parts:
\begin{itemize}

\item \emph{tracing}: from a JS-managed reference, find
  the JS-managed references immediately reachable from it, and

\item \emph{rooting}: find the JS-managed references which
  are reachable from the stack.

\end{itemize}
From these two functions, it is possible to find all of the
JS-managed references which are reachable from Rust. Together
with SpiderMonkey's regular GC, this allows the runtime to
find all of the reachable JS objects, and then to reclaim the
unreachable ones.

These interfaces are important for program correctness, since
under-approximation can result in use-after-free,
and over-approximation can result in space leaks.

In this section, we discuss how Josephine supports these interfaces.

\subsection{Tracing}
\label{sec:tracing}

Interfacing to the SpiderMonkey tracer via Josephine is achieved
in the same way as Servo~\cite{servo},
by implementing a trait:
\begin{verbatim}
  pub unsafe trait JSTraceable {
    unsafe fn trace(&self, trc: *mut JSTracer);
  }
\end{verbatim}
Josephine provides an implementation:
\begin{verbatim}
  unsafe impl<'a, C, T> JSTraceable for JSManaged<'a, C, T> where
    T: JSTraceable { ... }
\end{verbatim}
User-defined types can then implement the interface by
recursively visiting fields, for example:
\begin{verbatim}
  unsafe impl<'a, C, T> JSTraceable for NativeCell<'a, C> {
    unsafe fn trace(&self, trc: *mut JSTracer) {
      self.prev.trace(trc);
      self.next.trace(trc);
    }
  }
\end{verbatim}
This is a lot of unsafe boilerplate, but fortunately can
also be mechanized using meta-programming by marking a type
as \verb|#[derive(JSTraceable)]|.

One subtlety is that during tracing data of type $T$, the JS runtime
has a reference of type $\REF{}{T}$ given by the \verb|self| parameter
to \verb|trace|. For this to be safe, we have to ensure that there is
no mutable reference to that data. This is maintained by the
previously mentioned invariant:
\begin{quote}\em
  Any operation that can trigger garbage collection
  requires mutable access to the JS context.
\end{quote}
Tracing is triggered by garbage collection, and so had unique access
to the JS context, so there cannot be any other live mutable access
to any JS-managed data.

\subsection{Rooting}
\label{sec:rooting}

In languages with native support for GC, rooting is
supported by the compiler, which can provide metadata for
each stack frame allowing it to be traced. In languages like
Rust that do not have a native GC, this metadata is not
present, and instead rooting has to be performed explicitly.

This explicit rooting is needed whenever an object is
needed to outlive the borrow of the JS context that produced
it. For example, a function to insert a new cell after
an existing one is:
\begin{verbatim}
  pub fn insert<C, S>(cell: Cell<C>, data: String, cx: &mut JSContext<S>) where
    S: CanAccess + CanAlloc + InCompartment<C>,
    C: Compartment,
  {
    let old_next = cell.borrow(cx).next;
    let new_next = cx.manage(NativeCell {
      data: data,
      prev: Some(cell),
      next: old_next,
    });
    cell.borrow_mut(cx).next = Some(new_next);
    if let Some(old_next) = old_next {
      old_next.borrow_mut(cx).prev = Some(new_next);
    }
  }
\end{verbatim}
This is the ``code you would first think of'' for inserting an
element into a doubly-linked list, but is in fact not safe because
the local variables \verb|old_next| and \verb|new_next| have not been
rooted. If GC were triggered just after \verb|new_next| was created,
then it could be reclaimed, causing a later use-after-free.

Fortunately, Josephine will catch these safety problems, and report
errors such as:
\begin{verbatim}
  error[E0502]: cannot borrow `*cx` as mutable because
    it is also borrowed as immutable
    |
    |         let old_next = self.borrow(cx).next;
    |                                    -- immutable borrow occurs here
    |         let new_next = cx.manage(NativeCell {
    |                        ^^ mutable borrow occurs here
...
    |     }
    |     - immutable borrow ends here
\end{verbatim}
The fix is to explicitly root the local variables. In Josephine this is:
\begin{verbatim}
   let ref mut root1 = cx.new_root();
   let ref mut root2 = cx.new_root();
   let old_next = (... as before ...).in_root(root1);
   let new_next = (... as before ...).in_root(root2);
\end{verbatim}
The declaration of a \verb|root| allocates space on the stack
for a new root, and \verb|managed.in_root(root)| roots \verb|managed|.
Note that it is just the reference that is copied to the stack,
the JS-managed data itself doesn't move.
Roots have type $\JSRoot{\beta,T}$ where $\beta$ is the lifetime
of the root, and $T$ is the type being rooted.

Once the local variables are rooted, the code typecheck,
because rooting changes the lifetime of the JS-managed
data, for example (when $\alpha \subseteq \beta$):
\begin{quote}
  if~$p: \JSManaged{\alpha,C,T}$ \\
  and~$r: \REFMUT{\beta}{\JSRoot{\beta,\JSManaged{\beta,C,T[\beta/\alpha]}}}$ \\
  then~$p.\inRoot(r): \JSManaged{\beta,C,T[\beta/\alpha]}$.
\end{quote}
Before rooting, the JS-managed data had lifetime $\alpha$,
which is usually the lifetime of the borrow of the JS context
that created or accessed it.
After rooting, the JS-managed data has lifetime $\beta$,
which is the lifetime of the root itself. Since roots are
considered reachable by GC, the contents of a root
are guaranteed not to be GC'd during its lifetime,
so this rule is sound.

Note that this use of substitution $T[\beta/\alpha]$
is being used to extend the lifetime of the
JS-managed data, since $\alpha\subseteq\beta$. This
is in comparison to the use of substitution $T[\alpha/\beta]$
in \S\ref{sec:typing-access}, which was used to contract the
lifetime.

